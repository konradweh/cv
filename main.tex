\documentclass[11pt,a4paper]{article}

% Load all layout + style logic
% ============================================================
% PREAMBLE.TEX
% All non-content: packages, fonts, colors, layout, macros
% ============================================================

% -----------------------------
% Language
% -----------------------------
\usepackage[german]{babel}

% -----------------------------
% Page Layout / Graphics
% -----------------------------
\usepackage[a4paper,top=18mm,bottom=18mm,left=18mm,right=18mm]{geometry}
\usepackage{graphicx}
\usepackage{tikz}
\usepackage{paracol}
\usepackage{eso-pic}
\usepackage{calc}
\usepackage{array}
\usepackage{tabularx}

% -----------------------------
% Colors
% -----------------------------
\usepackage{xcolor}
\definecolor{cvlightblue}{HTML}{D6DCE4}
\definecolor{cvblue}{HTML}{AEB7C4}
\colorlet{cvgray}{black!65}
\colorlet{cvlightgray}{black!20}
\definecolor{cvblack}{HTML}{000000}

% -----------------------------
% Fonts (XeLaTeX / LuaLaTeX)
% -----------------------------
\usepackage{fontspec}
\defaultfontfeatures{Ligatures=TeX}

% Body font
\setmainfont{Calibri}
\setsansfont{Calibri}

% Headline font
\newfontfamily\CalibriLight{Calibri Light}

% -----------------------------
% Typography helpers
% -----------------------------
\usepackage{microtype}
\usepackage{lipsum}

% ============================================================
% Layout dimensions
% ============================================================

% Header-height
\newlength{\HeaderH}  \setlength{\HeaderH}{60mm}

% Photo dimensions
\newlength{\PhotoW}   \setlength{\PhotoW}{35mm}
\newlength{\PhotoH}   \setlength{\PhotoH}{45mm}
\newlength{\PhotoTopOffset}
\setlength{\PhotoTopOffset}{25mm}

% Column separation
\newlength{\ColSep}   \setlength{\ColSep}{12mm}
\newcommand{\LeftColRatio}{0.35} % <-- Adjust column ratio here

\newlength{\ContentW}
\newlength{\LeftColW}
\newlength{\RightColX}

\setlength{\ContentW}{\textwidth}
\addtolength{\ContentW}{-\ColSep}

\setlength{\LeftColW}{\LeftColRatio\ContentW}
\setlength{\RightColX}{\LeftColW}
\addtolength{\RightColX}{\ColSep}

% x-position of column separation line
\newlength{\SepLineX}
\setlength{\SepLineX}{\LeftColW}
\addtolength{\SepLineX}{0.5\ColSep}


\setlength{\parindent}{0pt}
\setlength{\parskip}{6pt}

% ============================================================
% Macros
% ============================================================

\newcommand{\cvsection}[1]{%
  \par
  {\CalibriLight
   \fontsize{11}{14}\selectfont
   \color{cvblack}
   \addfontfeatures{LetterSpace=25}
   \spaceskip=1em
   \MakeUppercase{#1}
   \par}%
   \vspace{1pt}
}

\newcommand{\cvfirstname}[1]{%
  {\CalibriLight
   \fontsize{27}{32}\selectfont
   \color{cvgray}
   \addfontfeatures{LetterSpace=25}%
   \spaceskip=1em
   \MakeUppercase{#1}}%
}

\newcommand{\cvlastname}[1]{%
  {\CalibriLight
   \fontsize{27}{32}\selectfont
   \color{cvblack}
   \addfontfeatures{LetterSpace=25}%
   \spaceskip=1em
   \MakeUppercase{#1}}%
}

\newcommand{\cvcv}[1]{%
  {\CalibriLight
   \fontsize{20}{24}\selectfont
   \color{cvgray}
   \addfontfeatures{LetterSpace=25}%
   \spaceskip=1em
   \MakeUppercase{#1}}%
}

% ============================================================
% Language skill bars
% ============================================================

% Styling knobs (easy to tweak)
\newlength{\LangBarDashW}   \setlength{\LangBarDashW}{7mm}   % width of one dash
\newlength{\LangBarDashH}   \setlength{\LangBarDashH}{0.8mm} % height of dash
\newlength{\LangBarGap}     \setlength{\LangBarGap}{2mm}     % gap between dashes
\newlength{\LangBarTotalW}                                   % total width of all dashes + gaps

\newlength{\LangGapTop}     \setlength{\LangGapTop}{8pt}   % space: title -> bar
\newlength{\LangGapBot}     \setlength{\LangGapBot}{8pt}   % space: bar -> description

\colorlet{cvbaron}{cvblue}        % "filled" color
\colorlet{cvbaroff}{cvlightgray}  % "empty" color

% Draw N dashes, first K are "on"
\newcommand{\cvlangbar}[2]{%
  \begin{tikzpicture}[baseline=(current bounding box.south)]
    \foreach \i in {1,...,#2} {%
      \pgfmathsetlengthmacro{\x}{(\i-1)*(\LangBarDashW+\LangBarGap)}%
      \ifnum\i>#1
        \fill[cvbaroff] (\x,0) rectangle ++(\LangBarDashW,\LangBarDashH);
      \else
        \fill[cvbaron]  (\x,0) rectangle ++(\LangBarDashW,\LangBarDashH);
      \fi
    }%
  \end{tikzpicture}%
}

% One language entry: name + level row, bar row, description row
% #1 Language label (e.g. Deutsch:)
% #2 Level (e.g. C2)
% #3 filled dashes (e.g. 5)
% #4 total dashes (e.g. 6)
% #5 description (e.g. Muttersprache)
\newcommand{\cvlanguage}[5]{%
  \par\addvspace{10pt}% space between language entries

  \begingroup
    \setlength{\parskip}{0pt}%

    % total width = total*dash + (total-1)*gap
    \setlength{\LangBarTotalW}{%
      \dimexpr #4\LangBarDashW + \numexpr#4-1\relax\LangBarGap \relax
    }%

    \noindent
    \vbox{\offinterlineskip
      % line 1
      \hbox to \LangBarTotalW{\textbf{#1}\hfil \textbf{#2}}%
      \kern\LangGapTop

      % bar
      \hbox{\cvlangbar{#3}{#4}}%
      \kern\LangGapBot

      % line 3
      \hbox to \LangBarTotalW{#5\hfil}%
    }%
  \endgroup
  \par
}

% ============================================================
% CV entry with a right-aligned date, vertically centered
% ============================================================

\newlength{\CVDateW}
\setlength{\CVDateW}{35mm} % <-- width reserved for the date column (adjust)

\newsavebox{\CVLeftBox}
\newlength{\CVLeftW}
\newlength{\CVLeftHT}

\newcommand{\cventry}[2]{%
  \par\addvspace{2pt}

  \begingroup
    \setlength{\parskip}{0pt}

    % Left width = full line minus date column
    \setlength{\CVLeftW}{\dimexpr\linewidth-\CVDateW\relax}

    % Typeset left block into a measuring box
    \sbox{\CVLeftBox}{%
      \begin{minipage}[t]{\CVLeftW}
        #1%
      \end{minipage}%
    }%

    % Measure total height of left block
    \setlength{\CVLeftHT}{\dimexpr\ht\CVLeftBox+\dp\CVLeftBox\relax}

    % Output row
    \noindent
    \usebox{\CVLeftBox}%
    \hfill
    \begin{minipage}[c][\CVLeftHT][c]{\CVDateW}
      \raggedleft{\color{cvlightblue}\small #2}%
    \end{minipage}%
  \endgroup
}

% ============================================================
% Header background + photo
% ============================================================

\AddToShipoutPictureBG{%
\begin{tikzpicture}[remember picture,overlay]

\fill[cvlightblue]
  (current page.north west) rectangle
  ([yshift=-\HeaderH]current page.north east);

\coordinate (PhotoNW) at ([xshift=18mm,yshift=-\PhotoTopOffset]current page.north west);
\node[anchor=north west, fill=white, inner sep=0pt] at (PhotoNW)
{\begin{minipage}[t][\PhotoH][c]{\PhotoW}
\centering
\includegraphics[width=\PhotoW,height=\PhotoH,keepaspectratio]{portrait.JPG}
\end{minipage}};

% vertical separation line
\draw[cvblue, line width=1.0pt]
  ([xshift=18mm+\SepLineX, yshift=-\HeaderH]current page.north west) --
  ([xshift=18mm+\SepLineX, yshift=18mm]current page.south west);

\end{tikzpicture}}

% --- Horizontal section separator (blue stripe) ---
% Draws a line across the full column width (from the middle divider to the outer edge)
\newcommand{\cvhstripe}{%
  \par\vspace{10pt}%
  \noindent{\color{cvblue}\rule{\linewidth}{0.8pt}}%
  \par\vspace{10pt}%
}

% ============================================================
% Body start offset
% ============================================================

\newlength{\BodyStart}
\setlength{\BodyStart}{\HeaderH}
\addtolength{\BodyStart}{-7mm}

% ============================================================
% Header text placement helper
% ============================================================

\newcommand{\PlaceHeaderText}[3]{%
  \smash{%
    \begin{tikzpicture}[remember picture,overlay]
      \node[anchor=north west, align=left] at
      ([xshift=18mm+\RightColX,yshift=-18mm-6mm]current page.north west)
      {%
        \cvfirstname{#1}\hspace{1.5em}\cvlastname{#2}\\[10pt]
        \cvcv{#3}%
      };
    \end{tikzpicture}%
  }%
  \par\nointerlineskip
}

\begin{document}
\pagestyle{empty}
\color{cvblack}

% Push content below the header + photo overlap
\vspace*{\BodyStart}

% -----------------------------
% Two-column layout
% -----------------------------
\columnratio{\LeftColRatio,\dimexpr 1\relax-\LeftColRatio\relax}
\setlength{\columnsep}{\ColSep}

% Header text inside the blue bar
\PlaceHeaderText{Konrad}{Wehkamp}{Curriculum Vitae}
\vspace*{0pt}

\begin{paracol}{2}
\noindent\strut

% =============================
% LEFT COLUMN
% =============================

\cvsection{Profile}
Master’s student in Physics and \newline
Technology for Space Applications \newline
with a strong focus on spacecraft \newline
systems, experimental testing, \newline
and hardware-oriented development \newline
in laboratory environments.


\cvhstripe


\cvsection{Language Skills}
\cvlanguage{German:}{C2}{6}{6}{native speaker}
\cvlanguage{English:}{C1}{5}{6}{fluent}
\cvlanguage{French:}{B1}{3}{6}{basic knowledge}


\cvhstripe


\cvsection{IT Skills}
\textbf{Python}: advanced
\par
\textbf{C++}: intermediate
\par
\textbf{LabVIEW}: intermediate
\par
\textbf{CAD (Inventor)}: intermediate
\par
\textbf{MS Office / LaTeX / Origin}: advanced
\par
\textbf{GitHub}: \href{https://github.com/konradweh}{github.com/konradweh}


\cvhstripe


\cvsection{Interests}
Creative projects (electronics, \newline
3D printing, woodworking)
\par
Field hockey and Bouldering


\cvhstripe


\cvsection{Volunteer Experience}
Coach and organizer of an inclusive \newline
hockey team for athletes with \newline
disabilities since 2021.
\par
C-level coaching license, responsibility \newline
for training and team development.


% =============================
% RIGHT COLUMN
% =============================
\switchcolumn
\noindent\strut

\cvsection{Education}\par\vspace*{8pt}

\cventry
  {Justus Liebig University Giessen}
  {%
      \cvitem{M.Sc. Physics and Technology for Space Applications}
      \cvlastitem{Focus on spacecraft propulsion, plasma physics, and space systems engineering}
  }
  {04.2025 -- present (parallel enrollment)}

\cventry
  {Justus Liebig University Giessen}
  {%
      \cvitem{B.Sc. Physics and Technology for Space Applications}
      \cvitem{Final grade: 1.9}
      \cvitem{Bachelor’s thesis: Emittance measurements on\newline
      reference ion sources for electric propulsion\newline
      characterization (Ref4EP project)}
      \cvlastitem{Hands-on work with ion sources, beam diagnostics, and performance evaluation methods}
  }
  {10.2021 -- 04.2026}

\cventry{Karl-Rehbein-Gymnasium Hanau}
  {%
      \cvitem{German university entrance qualification, grade: 1.7}
      \cvlastitem{Honors for outstanding achievement in Physics}
  }{07.2021}


\cvhstripe
\vspace*{1pt}


\cvsection{RELEVANT PROJECTS \& EXPERIENCE}

\cvsubsection{GSI Helmholtz Centre for Heavy Ion Research}

\cventry
  {Internship}
  {%
      \cvitem{Designed and integrated a cooling solution for \newline
      analogue electronics used in the SHIPTRAP}
      \cvitem{Supported the integration and operation of \newline
      experimental hardware in a high-vacuum and radiation-exposed environment}
      \cvlastitem{Performed hands-on assembly, maintenance, and troubleshooting of experimental components, including work on the electron gun at the target area}
  }
  {07.2025 -- 10.2025}


\vspace*{-4pt}


\cvsubsection{Ion Thruster Research Group – JLU Gießen}

\cventry{Student Researcher}
{%
  \cvitem{Conducted plasma measurements and diagnostics \newline
  in the context of electric propulsion research}
  \cvlastitem{Supported experimental investigations using\newline
  THz time-domain spectroscopy (THz-TDS)}
}{04.2023 -- 10.2023}

\cventry{Project Work}
{
  \cvitem{Developed a global Python-based model for\newline
  multi-species plasmas in a small team}
  \cvlastitem{Used the model to study plasma behavior relevant to electric propulsion systems}
}{10.2024 -- 04.2026}

\cventry{Project Work}
{
  \cvitem{Developed a Python-based simulation of \newline
  atmospheric reentry for different space vehicles}
  \cvlastitem{Modeled key physical effects including aerodynamic forces, thermal loads, and flight dynamics}
}{10.2025 -- 04.2026}

\end{paracol}

\newpage

\newgeometry{top=10mm,bottom=22mm,left=20mm,right=20mm}
\setstretch{1.08}
\setlength{\parskip}{5pt}

% Header text inside the blue bar
\PlaceHeaderText{Konrad}{Wehkamp}{Motivation Letter}
\vspace*{0pt}

\vspace*{\BodyStart}
\noindent\strut



\cvsection{Motivation – Internship Application}

I am applying for an internship at ClearSpace because I am strongly motivated to contribute to technologies that enable a sustainable and responsible use of Earth’s orbital environment. Active debris removal addresses one of the most critical challenges in modern spaceflight, and ClearSpace’s mission-oriented, engineering-driven approach aligns closely with both my academic background and my personal motivation to work on hardware that has a direct operational impact.

I am currently a Master’s student in Physics and Technology for Space Applications at Justus Liebig University Giessen. My studies focus on space systems engineering with a strong emphasis on propulsion and spacecraft subsystems, covering topics from plasma physics and electric propulsion to system-level design considerations. While my primary specialization has been in electric propulsion, my academic training and project work have consistently connected physical modeling with practical engineering constraints.

Through my Bachelor’s thesis on the development and experimental investigation of a reference ion source for ion thruster characterization, I gained hands-on experience in designing and operating experimental setups, working with diagnostics, and evaluating performance-relevant data. In parallel, my work on global plasma simulations and an atmospheric reentry model in Python strengthened my ability to translate physical models into practical analysis tools that support engineering decisions.

During my internship at the GSI Helmholtz Centre for Heavy Ion Research, I worked in a complex experimental environment where I designed a cooling system for sensitive analogue components and performed hands-on maintenance and repair work on an electron gun in the target area. This experience strengthened my understanding of hardware-oriented development, integration constraints, and the responsibility associated with maintaining reliable operation of mission-critical subsystems within an interdisciplinary team.

For an internship at ClearSpace, I am particularly interested in contributing to spacecraft systems engineering, hardware development, and integration and testing activities. I am motivated to work close to hardware, to understand subsystem interactions, and to support verification and validation efforts that are essential for safe rendezvous, capture, and deorbit missions. I would welcome the opportunity to contribute to defined technical tasks within an ongoing project, while further developing my practical engineering skills in a spaceflight context.

\vspace{0.5em}

I am available for a full-time, in-person internship starting in April 2026, with a minimum duration of four months. I am flexible regarding the exact start date and would be open to a longer internship period if beneficial. I would be happy to provide further information if needed.


\end{document}